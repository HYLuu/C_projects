\documentclass[12pt]{article}
\usepackage{fullpage,fourier,booktabs,amsmath}
\usepackage[english]{babel}
\usepackage[autostyle, english=american]{csquotes}
\MakeOuterQuote{"}
\title{Design Document - Assignment 2}
\author{Hsiang Yun Lu}
\begin{document}\maketitle

\section{Purpose of Programs}

There are seven \texttt{C} programs. Six of them are implementation of different mathematical formulas which are used to compute the fundamental constants \textit{e} and $\pi$. The \texttt{mathlib-test.c} is a test harness that compares the created implemented functions with that of the math library’s.

\section{Files To Be Included in Directory \textit{asgn2}}

\begin{itemize}
   \item \texttt{newton.c}
   \begin{itemize}
     \item This file contains two functions. One implements the square root approximation using Newton’s method and the other returns the number of computed iterations
   \end{itemize}
   \item \texttt{e.c}
   \begin{itemize}
     \item This file contains two functions. One implements the Taylor series to approximate Euler’s number \textit{e} and the other returns the number of computed terms
   \end{itemize}
   \item \texttt{madhava.c}
   \begin{itemize}
     \item This file contains two functions. One implements the Madhava series to approximate $\pi$ and the other returns the number of computed terms
   \end{itemize}
   \item \texttt{euler.c}
   \begin{itemize}
     \item This file contains two functions. One implements the Euler’s solution to approximate $\pi$ and the other returns the number of computed terms.
   \end{itemize}
   \item \texttt{bbp.c}
   \begin{itemize}
     \item This file contains two functions. One implements the Bailey-Borwein-Plouffe formula to approximate $\pi$ and the other returns the number of computed terms.
   \end{itemize}
   \item \texttt{viete.c}
   \begin{itemize}
     \item This file contains two functions. One implements the Viète’s formula to approximate $\pi$ and the other returns the number of computed factors.
   \end{itemize}
   \item \texttt{mathlib-test.c}
   \begin{itemize}
     \item The program with the main() function which tests each of created math library functions by allowing command-line options
   \end{itemize}
   \item \texttt{mathlib.h}
   \begin{itemize}
     \item The header file that contains the interface of the created math library (the above programs) 
   \end{itemize}
   \item \texttt{Makefile}
   \begin{itemize}
     \item This file taht compiles the math library programs and builds the \texttt{mathlib-test} executable
   \end{itemize}
   \item \texttt{README.md}
   \begin{itemize}
     \item This file describes how to use all the programs and \texttt{Makefile}, including explanations on command-line options
   \end{itemize}
   \item \texttt{DESIGN.pdf}
   \begin{itemize}
     \item This file describes the design and design process for all the proigrams
     \item Include pseudocode
   \end{itemize}
   \item \texttt{WRITEUP.pdf}
   \begin{itemize}
     \item Include the graphs that displays the difference between the values reported by the created math functions and that of the math library’s
     \item Include analysis and explanations for any discrepancies and findings on the testing of created math library
   \end{itemize}
 \end{itemize}

\section{Design Process}

Since there is a function to return the number of computed terms in each small program, I set the \texttt{static} counter in the beginning. According to each formula, different $x$ or $y$ variables are defined to represent $k$ in the formulas or the previous value of power $k-1$. The calculation in the WHILE loop will continue until the difference between previous value (last) and current value (current) is smaller than EPSILON. \\
\\
For the \texttt{mathlib-test.c} program, I first create function for each small approximation program. In each function, an integer \textit{k} is used to control whether the function prints only the line of number of iterations. A \textit{usage} function is created and used to print program synopsis and usage. The \textit{main} function first checks the arguments. It prints program synopsis and usage if there is no argument. I then use a integer \textit{present} to represent the existence of argument "-s". If "-s" exists, all the \textit{k} will be set to 1 and will print the requested test results and their numbers of iterations. Otherwise, all the \textit{k} will be set to 1 and print only the requested test result lines. \\

\section{Design/Structure}

(1) \texttt{newton.c} \\
\\
SET counter to 0 \\
\\
function sqrt\_newton(x) \\
\indent SET $s \leftarrow 1$ , $y \leftarrow 1$, $z \leftarrow 0$ \\
\indent WHILE $x > 4$ do \\
\indent \indent $x \leftarrow x / 4$ \\
\indent \indent $s \leftarrow s \times 2$ \\
\indent END WHILE \\
\indent WHILE $| y - z | >$ EPSILON do \\
\indent \indent $z \leftarrow y$ \\
\indent \indent $y \leftarrow \frac{1}{2} (y + \frac{x}{y})$  \\
\indent \indent counter $\leftarrow$ counter + 1 \\
\indent END WHILE \\
\indent RETURN s $\times$ y \\
END function \\
\\
\\
function sqrt\_newton\_iters() \\
\indent RETURN counter \\
END function \\
\\
\\
(2) \texttt{e.c} \\
\\
SET counter to 0 \\
\\
function e() \\
\indent SET $x \leftarrow 1$ , $y \leftarrow 1$, $last \leftarrow 0$, $current \leftarrow 1$ \\
\indent WHILE $| current - last | >$ EPSILON do \\
\indent \indent last $\leftarrow$ current \\
\indent \indent current $\leftarrow$ current + $\frac{1}{x \times y}$ \\
\indent \indent y $\leftarrow x \times y$ \\
\indent \indent x $\leftarrow$ x + 1 \\
\indent \indent counter $\leftarrow$ counter + 1 \\
\indent END WHILE \\
\indent RETURN current \\
END function \\
\\
\\
function e\_terms() \\
\indent RETURN counter \\
END function \\
\\
\\
(3) \texttt{madhava.c} \\
\\
SET counter to 0 \\
\\
function pi\_madhava() \\
\indent SET $x \leftarrow 1$ , $y \leftarrow 1$, $last \leftarrow 0$, $current \leftarrow 1$ \\
\indent WHILE $| current - last | >$ EPSILON do \\
\indent \indent last $\leftarrow$ current \\
\indent \indent current $\leftarrow$ current + $\frac{1}{2 \times x + 1} \times y \times \frac{1}{-3}$ \\
\indent \indent y $\leftarrow y \times \frac{1}{-3}$ \\
\indent \indent x $\leftarrow$ x + 1 \\
\indent \indent counter $\leftarrow$ counter + 1 \\
\indent END WHILE \\
\indent RETURN current $\times \sqrt{12}$ \\
END function \\
\\
\\
function pi\_madhava\_terms() \\
\indent RETURN counter \\
END function \\
\\
\\
(4) \texttt{euler.c} \\
\\
SET counter to 0 \\
\\
function pi\_euler() \\
\indent SET $x \leftarrow 2$ , $last \leftarrow 0$, $current \leftarrow 1$ \\
\indent WHILE $| current - last | >$ EPSILON do \\
\indent \indent last $\leftarrow$ current \\
\indent \indent current $\leftarrow$ current + $\frac{1}{x \times x} $ \\
\indent \indent x $\leftarrow$ x + 1 \\
\indent \indent counter $\leftarrow$ counter + 1 \\
\indent END WHILE \\
\indent RETURN $\sqrt{current \times 6}$ \\
END function \\
\\
\\
function pi\_euler\_terms() \\
\indent RETURN counter \\
END function \\
\\
\\
(5) \texttt{bbp.c} \\
\\
SET counter to 0 \\
\\
function pi\_bbp() \\
\indent SET $x \leftarrow 1$ , $y \leftarrow 1$, $last \leftarrow 0$, $current \leftarrow 4 - \frac{1}{2} - \frac{1}{5} - \frac{1}{6}$ \\
\indent WHILE $| current - last | >$ EPSILON do \\
\indent \indent last $\leftarrow$ current \\
\indent \indent current $\leftarrow$ current + $y \times \frac{1}{16} \times (\frac{4}{8x + 1} - \frac{2}{8x + 4} - \frac{1}{8x + 5}- \frac{1}{8x + 6})$ \\
\indent \indent y $\leftarrow y \times \frac{1}{16}$ \\
\indent \indent x $\leftarrow$ x + 1 \\
\indent \indent counter $\leftarrow$ counter + 1 \\
\indent END WHILE \\
\indent RETURN current \\
END function \\
\\
\\
function pi\_bbp\_terms() \\
\indent RETURN counter \\
END function \\
\\
\\
(6) \texttt{viete.c} \\
\\
SET counter to 0 \\
\\
function pi\_viete() \\
\indent SET $x \leftarrow \sqrt{2}$ , $last \leftarrow 0$, $current \leftarrow \frac{\sqrt{2}}{2}$ \\
\indent WHILE $| current - last | >$ EPSILON do \\
\indent \indent last $\leftarrow$ current \\
\indent \indent current $\leftarrow$ current $\times \frac{\sqrt{2 + x}}{2}$ \\
\indent \indent x $\leftarrow \sqrt{2 + x}$ \\
\indent \indent counter $\leftarrow$ counter + 1 \\
\indent END WHILE \\
\indent RETURN $\frac{1}{\frac{current}{2}}$ \\
END function \\
\\
\\
function pi\_viete\_factors() \\
\indent RETURN counter \\
END function \\
\\
\\
(7) \texttt{mathlib-test.c} \\
\\
DEFINE command-line options "aebmrvnsh" \\
\\
function usage(program) \\
\indent print(program synopsis and usage) \\
END function \\
\\
\\
M\_PI $\leftarrow$ $\pi$ constant value in \textit{math.h} \\
M\_E $\leftarrow$ \textit{e} constant value in \textit{math.h} \\
\\
function get\_e(k) \\
\indent IF k = 1 \\
\indent \indent PRINT e() output, M\_E, difference \\
\indent ELSE \\
\indent \indent PRINT e() output, M\_E, difference \\
\indent \indent PRINT e\_terms() output \\
END function \\
\\
\\
function get\_m(k) \\
\indent IF k = 0 \\
\indent \indent PRINT madhava() output, M\_PI, difference \\
\indent ELSE \\
\indent \indent PRINT madhava() output, M\_PI, difference \\
\indent \indent PRINT madhava\_terms() output \\
END function \\
\\
\\
function get\_r(k) \\
\indent IF k = 0 \\
\indent \indent PRINT euler() output, M\_PI, difference \\
\indent ELSE \\
\indent \indent PRINT euler() output, M\_PI, difference \\
\indent \indent PRINT euler\_terms() output \\
END function \\
\\
\\
function get\_r(k) \\
\indent IF k = 0 \\
\indent \indent PRINT euler() output, M\_PI, difference \\
\indent ELSE \\
\indent \indent PRINT euler() output, M\_PI, difference \\
\indent \indent PRINT euler\_terms() output \\
END function \\
\\
\\
function get\_b(k) \\
\indent IF k = 0 \\
\indent \indent PRINT bbp() output, M\_PI, difference \\
\indent ELSE \\
\indent \indent PRINT bbp() output, M\_PI, difference \\
\indent \indent PRINT bbp\_terms() output \\
END function \\
\\
\\
function get\_v(k) \\
\indent IF k = 0 \\
\indent \indent PRINT viete() output, M\_PI, difference \\
\indent ELSE \\
\indent \indent PRINT viete() output, M\_PI, difference \\
\indent \indent PRINT viete\_terms() output \\
END function \\
\\
\\
function get\_n(k) \\
\indent IF k = 0 \\
\indent \indent FOR i = 1 to 10, move 0.1 each time \\
\indent \indent \indent PRINT sqrt\_newton() output, sqrt() output, difference \\
\indent \indent END FOR \\
\indent ELSE \\
\indent \indent FOR i = 1 to 10, move 0.1 each time \\
\indent \indent \indent PRINT sqrt\_newton() output, sqrt() output, difference \\
\indent \indent \indent PRINT viete\_terms() output \\
\indent END IF \\
END function \\
\\
\\
function main(argument) \\
\indent SET present $\leftarrow 0$ \\
\\
\indent IF no argument \\
\indent \indent usage() \\
\indent END IF \\
\\
\indent FOR each read-in argument \\
\indent \indent IF argument == "-s" \\
\indent \indent \indent present $\leftarrow 1$ \\
\indent \indent END IF \\
\indent END FOR \\
\\
\indent IF present == 1 \\
\indent \indent WHILE there is at least an argument \\
\indent \indent \indent SWITCH argument \\
\indent \indent \indent \indent case argument "e" do get\_e(1) \\
\indent \indent \indent \indent case argument "r" do get\_r(1) \\
\indent \indent \indent \indent case argument "b" do get\_b(1) \\
\indent \indent \indent \indent case argument "m" do get\_m(1) \\
\indent \indent \indent \indent case argument "v" do get\_v(1) \\
\indent \indent \indent \indent case argument "n" do get\_n(1) \\
\indent \indent \indent \indent case argument "a" do \\
\indent \indent \indent \indent \indent get\_e(1) \\
\indent \indent \indent \indent \indent get\_r(1) \\
\indent \indent \indent \indent \indent get\_b(1) \\
\indent \indent \indent \indent \indent get\_m(1) \\
\indent \indent \indent \indent \indent get\_v(1) \\
\indent \indent \indent \indent \indent get\_n(1) \\
\indent \indent \indent \indent case argument "s" do \\
\indent \indent \indent \indent \indent IF there is only one argument \\
\indent \indent \indent \indent \indent \indent usage() \\
\indent \indent \indent \indent \indent ELSE \\
\indent \indent \indent \indent \indent \indent break \\
\indent \indent \indent \indent \indent END IF \\
\indent \indent \indent \indent case argument "h" do usage() \\
\indent \indent \indent \indent default do usage() \\
\indent \indent \indent \indent \indent return EXIT\_FAILURE \\
\indent \indent \indent END SWITCH \\
\indent \indent END WHILE \\
\indent ELSE \\
\indent \indent WHILE there is at least an argument \\
\indent \indent \indent SWITCH argument \\
\indent \indent \indent \indent case argument "e" do get\_e(0) \\
\indent \indent \indent \indent case argument "r" do get\_r(0) \\
\indent \indent \indent \indent case argument "b" do get\_b(0) \\
\indent \indent \indent \indent case argument "m" do get\_m(0) \\
\indent \indent \indent \indent case argument "v" do get\_v(0) \\
\indent \indent \indent \indent case argument "n" do get\_n(0) \\
\indent \indent \indent \indent case argument "a" do \\
\indent \indent \indent \indent \indent get\_e(0) \\
\indent \indent \indent \indent \indent get\_r(0) \\
\indent \indent \indent \indent \indent get\_b(0) \\
\indent \indent \indent \indent \indent get\_m(0) \\
\indent \indent \indent \indent \indent get\_v(0) \\
\indent \indent \indent \indent \indent get\_n(0) \\
\indent \indent \indent \indent case argument "s" do \\
\indent \indent \indent \indent \indent IF there is only one argument \\
\indent \indent \indent \indent \indent \indent usage() \\
\indent \indent \indent \indent \indent ELSE \\
\indent \indent \indent \indent \indent \indent break \\
\indent \indent \indent \indent \indent END IF \\
\indent \indent \indent \indent case argument "h" do usage() \\
\indent \indent \indent \indent default do usage() \\
\indent \indent \indent \indent \indent return EXIT\_FAILURE \\
\indent \indent \indent END SWITCH \\
\indent \indent END WHILE \\
\indent END IF \\
END function \\
\\
\section{Credit}

\begin{itemize}
  \item Implementation in \texttt{newton.c} was inspired and rewrited from the python code in the assignment specification pdf, and the \texttt{indirect.c} program provided by Dr. Long
  \item The command-line option use took reference from \texttt{monte\_carlo.c} program
\end{itemize}
\end{document}