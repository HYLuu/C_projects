\documentclass[12pt]{article}
\usepackage{fullpage,fourier,booktabs,amsmath, hyperref}
\usepackage[english]{babel}
\usepackage[autostyle, english=american]{csquotes}
\MakeOuterQuote{"}
\title{Design Document - Assignment 4}
\author{Hsiang Yun Lu}
\begin{document}\maketitle

\section{Purpose of Programs}

The \texttt{life.c} program implements the Game of Life evolution based on input files or STDIN and shows the animation of the evolution of the universe. The \texttt{universe.c} program includes the structure of \texttt{Universe} and the accessing and manipulating functions that will be used in \texttt{life.c}.

\section{Files To Be Included in Directory \textit{asgn3}}

\begin{itemize}
   \item \texttt{universe.c}
   \begin{itemize}
     \item This is the implementation of the \texttt{Universe} ADT, including type construction and associated functions.
   \end{itemize}
   \item \texttt{universe.h}
   \begin{itemize}
     \item This file specifies the interface to the \texttt{Universe} ADT, including declaration of new \texttt{type} and \texttt{universe}.
   \end{itemize}
   \item \texttt{life.c}
   \begin{itemize}
     \item This program contains \texttt{main()} and can read in inputs to create the universe. It's the main implementation of the Game of Life.
   \end{itemize}
   \item \texttt{Makefile}
   \begin{itemize}
     \item This file compiles the sorting programs and builds the \texttt{life} executable
   \end{itemize}
   \item \texttt{README.md}
   \begin{itemize}
     \item This file describes how to use all the programs and \texttt{Makefile}, including explanations on command-line options
   \end{itemize}
   \item \texttt{DESIGN.pdf}
   \begin{itemize}
     \item This file describes the design and design process for all the programs
     \item Include pseudocode
   \end{itemize}
   \item \texttt{WRITEUP.pdf}
   \begin{itemize}
     \item Things learned in this assignment in detail
   \end{itemize}
 \end{itemize}

\section{Design Process of \texttt{universe.c}}

The \texttt{universe.c} includes the detailed of \texttt{Universe } struct and all the helper, accessor, constructor, destructor and manipulator functions for \texttt{Universe}. All of the "uv-" prefixed functions were implemented according to the descriptions in the assignment PDF. I added a \texttt{in\_bound} function to check whether the given row/column pairs are out-of-bounds and a \texttt{swap} function to help swap the pointers of two \texttt{Universe}. The other helper function is \texttt{one\_generation}, which performs one generation of the cell evolution according to the given rules. All these functions are used in \texttt{life.c}.

\section{Design/Structure of \texttt{universe.c}}

DEFINE Universe instance $\leftarrow$ [rows, cols, **grid, toroidal] \\
\\
function int in\_bound(uint32\_t rows, uint32\_t cols, uint32\_t r, uint32\_t c) \\
\indent IF (r+1) <= rows and (c+1) <= cols \\
\indent \indent RETURN 1 \\
\indent ELSE \\
\indent \indent RETURN 0 \\
\indent END IF \\
END function \\
\\
\\
function void swap(Universe *x, Universe *y) \\
\indent tmp $\leftarrow$ *x \\
\indent *x $\leftarrow$ *y \\
\indent *y $\leftarrow$ tmp \\
END function \\
\\
\\
function int32\_t mod(int32\_t a, int32\_t b) \\
\indent n $\leftarrow$ a \% b \\
\indent RETURN (n + b) if n < 0 else RETURN n \\
END function \\
\\
\\
function Universe *uv\_create(uint32\_t rows, uint32\_t cols, bool toroidal) \\
\indent SET Universe *u \\
\indent u $\leftarrow$ calloc(1, sizeof(Universe)) \\
\indent SET **grid $\leftarrow$ calloc(rows, sizeof(bool pointer)) \\
\indent FOR r = 0 to rows \\
\indent \indent grid[r] $\leftarrow$ calloc(cols, sizeof(bool)) \\
\indent END FOR \\
\indent u->rows $\leftarrow$ rows \\
\indent u->cols $\leftarrow$ cols \\
\indent u->grid $\leftarrow$ grid \\
\indent u->toroidal $\leftarrow$ toroidal \\
\indent RETURN universe \\
END function \\
\\
\\
function void uv\_delete(Universe *u) \\
\indent FOR i = 0 to u->rows \\
\indent \indent free(u->grid[i]) \\
\indent free(u->grid) \\
\indent free(u) \\
END function \\
\\
\\
function uint32\_t uv\_rows(Universe *u) \\
\indent RETURN u->rows \\
END function \\
\\
\\
function uint32\_t uv\_cols(Universe *u) \\
\indent RETURN u->cols \\
END function \\
\\
\\
function void uv\_live\_cell(Universe *u, uint32\_t r, uint32\_t c) \\
\indent IF (r+1) <= u->rows and (c+1) <= u->cols \\
\indent \indent u->grid[r][c] $\leftarrow$ TRUE \\
\indent END IF \\
END function \\
\\
\\
function void uv\_dead\_cell(Universe *u, uint32\_t r, uint32\_t c) \\
\indent IF (r+1) <= u->rows and (c+1) <= u->cols \\
\indent \indent u->grid[r][c] $\leftarrow$ FALSE \\
\indent END IF \\
END function \\
\\
\\
function bool uv\_get\_cell(Universe *u, uint32\_t r, uint32\_t c) \\
\indent IF (r+1) <= u->rows and (c+1) <= u->cols \\
\indent \indent RETURN u->grid[r][c] \\
\indent ELSE \\
\indent \indent RETURN FALSE \\
\indent END IF \\
END function \\
\\
\\
function bool uv\_populate(Universe *u, FILE *infile) \\
\indent SET r $\leftarrow$ 0 \\
\indent SET c $\leftarrow$ 0 \\
\indent WHILE TRUE \\
\indent \indent SET input $\leftarrow$ fscanf(infile, format read in 2 uint32\_t, $&r$, $&c$) \\
\indent \indent IF input == 2 and in\_bound(u->rows, u->cols, r, c) == 1 \\
\indent \indent \indent SET u->grid[r][c] $\leftarrow$ TRUE \\
\indent \indent ELSE IF in\_bound(u->rows, u->cols, r, c) == 0 \\
\indent \indent \indent RETURN FALSE \\
\indent \indent ELSE IF read till the end of file \\
\indent \indent \indent RETURN TRUE \\
\indent \indent \indent break \\
\indent \indent ELSE IF input not equal 2 and input not equal end of file \\
\indent \indent \indent RETURN FALSE \\
\indent \indent END IF \\
\indent END WHILE \\
END function \\
\\
\\
function uint32\_t uv\_census(Universe *u, uint32\_t r, uint32\_t c) \\
\indent IF u->toroidal == false \\
\indent \indent SET live $\leftarrow$ 0 \\
\indent r\_int $\leftarrow$ (int32\_t) r \\
\indent c\_int $\leftarrow$ (int32\_t) c \\
\indent \indent FOR i = r\_int-1 to r\_int+1 \\
\indent \indent \indent FOR j = c\_int-1 to c\_int+1 \\
\indent \indent \indent \indent IF (i+1) <= u->rows and (j+1) <= u->cols and i >= 0 and j >= 0 and u->grid[i][j] == TRUE \\
\indent \indent \indent \indent \indent live $\leftarrow$ live + 1 \\
\indent \indent \indent \indent END IF \\
\indent \indent \indent END FOR \\
\indent \indent END FOR \\
\indent \indent IF u->grid[r][c] == TRUE \\
\indent \indent \indent RETURN live - 1 \\
\indent \indent ELSE \\
\indent \indent \indent RETURN live \\
\indent \indent END IF \\
\indent ELSE \\
\indent \indent SET live $\leftarrow$ 0 \\
\indent r\_int $\leftarrow$ (int32\_t) r \\
\indent c\_int $\leftarrow$ (int32\_t) c \\
\indent \indent FOR i = r\_int-1 to r\_int+1 \\
\indent \indent \indent FOR j = c\_int-1 to c\_int+1 \\
\indent \indent \indent \indent IF (i+1) > u->rows or i < 0 or (j + 1) > u->cols or j < 0 \\
\indent \indent \indent \indent \indent IF u->grid[mod(i, u->rows)][mod(j, u->cols)] == TRUE \\
\indent \indent \indent \indent \indent \indent live $\leftarrow$ live + 1 \\
\indent \indent \indent \indent \indent END IF \\
\indent \indent \indent \indent ELSE \\
\indent \indent \indent \indent \indent IF u->grid[i][j] == TRUE \\
\indent \indent \indent \indent \indent \indent live $\leftarrow$ live + 1 \\
\indent \indent \indent \indent \indent END IF \\
\indent \indent \indent \indent END IF \\
\indent \indent \indent END FOR \\
\indent \indent END FOR \\
\indent \indent IF u->grid[r][c] == TRUE \\
\indent \indent \indent RETURN live - 1 \\
\indent \indent ELSE \\
\indent \indent \indent RETURN live \\
\indent \indent END IF \\
\indent END IF \\
END function \\
\\
\\
function void uv\_print(Universe *u, FILE *outfile) \\
\indent FOR i = 0 to u->rows \\
\indent \indent FOR j = 0 to u->cols \\
\indent \indent \indent IF (j + 1) == u->cols \\
\indent \indent \indent \indent IF u->grid[i][j] == TRUE \\
\indent \indent \indent \indent \indent PRINT "o\n" to outfile \\
\indent \indent \indent \indent ELSE \\
\indent \indent \indent \indent \indent PRINT ".\n" to outfile \\
\indent \indent \indent \indent END IF \\
\indent \indent \indent ELSE \\
\indent \indent \indent \indent IF u->grid[i][j] == TRUE \\
\indent \indent \indent \indent \indent PRINT "o" to outfile \\
\indent \indent \indent \indent ELSE \\
\indent \indent \indent \indent \indent PRINT "." to outfile \\
\indent \indent \indent \indent END IF \\
\indent \indent \indent END IF \\
\indent \indent END FOR \\
\indent END FOR \\
END function \\
\\
\\
function void one\_generation(Universe *u\_A, Universe *u\_B, uint32\_t r, uint32\_t c) \\
\indent FOR i = 0 to r \\
\indent \indent FOR j = 0 to c \\
\indent \indent \indent IF uv\_get\_cell(u\_A, i, j) == TRUE and uv\_census(u\_A, i, j) == 2 \\
\indent \indent \indent \indent uv\_live\_cell(u\_B, i, j) \\
\indent \indent \indent ELSE IF uv\_get\_cell(u\_A, i, j) == TRUE and uv\_census(u\_A, i, j) == 3 \\
\indent \indent \indent \indent uv\_live\_cell(u\_B, i, j) \\
\indent \indent \indent ELSE IF uv\_get\_cell(u\_A, i, j) == FALSE and uv\_census(u\_A, i, j) == 3 \\
\indent \indent \indent \indent uv\_live\_cell(u\_B, i, j) \\
\indent \indent \indent ELSE \\
\indent \indent \indent \indent uv\_dead\_cell(u\_B, i, j) \\
\indent \indent \indent END IF \\
\indent \indent END FOR \\
\indent END FOR \\
END function \\
\\
\\
\section{Design Process of \texttt{life.c}}

This program first reads the command-line options and the input files or STDIN, creates and populates two \texttt{Universe}, performs some generation of cell evolution, displays the evolution animation on screen, and finally outputs the final state of the \texttt{Universe} to a file or STDOUT. I first declared the helper functions I wrote in \texttt{universe.c} in order to use them in \texttt{life.c}. Then I wrote a function for showing helping information before \texttt{main()}.

In \texttt{main()}, I declared variables for storing command-line arguments and input/output file names. After reading the command-line arguments, the first line of the input was read and the numbers were stored for creating \texttt{Universe}. The input data were used to create and populate two \texttt{Universe} using functions in \texttt{universe.c}. The \texttt{FILE} pointer for input stream should be closed after the two \texttt{Universe} were successfully populated. Based on the command-line arguments, for each generation display \texttt{Universe A}, perform one generation and then swap the pointers of two \texttt{Universe} if "-s" was not specified. After displaying animation, print out the final state of the \texttt{Universe}. If "-s" was specified, print out the final state of the \texttt{Universe} without animation. Finally, I deleted both \texttt{Universe} after printing successfully.

\section{Design/Structure of \texttt{life.c}}

DEFINE command-line options "tsn:i:o:h" \\
\\
function usage(program) \\
\indent print(program synopsis and usage) \\
END function \\
\\
\\
function main(arguments) \\
\indent SET opt $\leftarrow$ 0 \\
\indent SET torus $\leftarrow$ FALSE \\
\indent SET silence $\leftarrow$ FALSE \\
\indent SET num $\leftarrow$ 100 \\
\indent SET inp $\leftarrow$ FALSE \\
\indent SET out $\leftarrow$ FALSE \\
\indent SET char *input \\
\indent SET char *output \\
\indent SET FILE *finptr \\
\indent SET FILE *foutptr \\
\indent SET r $\leftarrow$ 0 \\
\indent SET c $\leftarrow$ 0 \\
\\
\indent WHILE opt $\leftarrow$ here is at least an argument \\
\indent \indent SWITCH(opt) \\
\indent \indent \indent case argument "t" SET torus $\leftarrow$ TRUE \\
\indent \indent \indent \indent break \\
\indent \indent \indent case argument "s" SET silence $\leftarrow$ TRUE \\
\indent \indent \indent \indent break \\
\indent \indent \indent case argument "n" SET num $\leftarrow$ number converted from read-in string \\
\indent \indent \indent \indent break \\
\indent \indent \indent case argument "i" \\
\indent \indent \indent \indent inp $\leftarrow$ TRUE \\
\indent \indent \indent \indent input $\leftarrow$ optarg \\
\indent \indent \indent \indent break \\
\indent \indent \indent case argument "o" \\
\indent \indent \indent \indent out $\leftarrow$ TRUE \\
\indent \indent \indent \indent output $\leftarrow$ optarg \\
\indent \indent \indent \indent break \\
\indent \indent \indent case argument "h" do usage(argv[0]) \\
\indent \indent \indent \indent EXIT \\
\indent \indent \indent default do usage(argv[0]) \\
\indent \indent \indent \indent return EXIT\_FAILURE \\
\indent \indent END SWITCH \\
\indent END WHILE \\
\\
\indent IF inp == FALSE \\
\indent \indent fscanf(stdin, format read in 2 uint32\_t, $\&r$, $\&c$) \\
\indent ELSE \\
\indent \indent IF (finptr $\leftarrow$ fopen(input, "r")) does not exist \\
\indent \indent \indent PRINT ERROR \\
\indent \indent \indent EXIT \\
\indent \indent ELSE \\
\indent \indent \indent fscanf(finptr, format read in 2 uint32\_t, $\&r$, $\&c$) \\
\indent \indent END IF \\
\indent END IF \\
\\
\indent SET Universe *uni\_A $\leftarrow$ uv\_create(r, c, torus) \\
\indent SET Universe *uni\_A $\leftarrow$ uv\_create(r, c, torus) \\
\\
\indent IF inp == FALSE \\
\indent \indent populated = uv\_populate(uni\_A, stdin) \\
\indent \indent IF populated == FALSE \\
\indent \indent \indent PRINT ERROR \\
\indent \indent \indent EXIT \\
\indent \indent END IF \\
\indent ELSE \\
\indent \indent populated = uv\_populate(uni\_A, finptr) \\
\indent \indent IF populated == FALSE \\
\indent \indent \indent PRINT ERROR \\
\indent \indent \indent EXIT \\
\indent \indent END IF \\
\indent END IF \\
\\
\indent CLOSE finptr \\
\\
\indent SET r\_A $\leftarrow$ uv\_rows(uni\_A) \\
\indent SET c\_A $\leftarrow$ uv\_cols(uni\_A) \\
\\
\indent IF silence == FALSE \\
\indent \indent initscr() \\
\indent \indent curs\_set(FALSE) \\
\indent \indent FOR k = 0 to num-1 \\
\indent \indent \indent clear() \\
\indent \indent \indent FOR i = 0 to r\_A-1 \\
\indent \indent \indent \indent FOR j = 0 to c\_A-1 \\
\indent \indent \indent \indent \indent IF uv\_get\_cell(uni\_A, i, j) == TRUE \\
\indent \indent \indent \indent \indent \indent PRINT "o" at location (i, j) on screen \\
\indent \indent \indent \indent \indent ELSE \\
\indent \indent \indent \indent \indent \indent PRINT "." at location (i, j) on screen \\
\indent \indent \indent \indent \indent END IF \\
\indent \indent \indent \indent END FOR \\
\indent \indent \indent END FOR \\
\indent \indent \indent refresh() \\
\indent \indent \indent usleep(50000) \\
\indent \indent \indent one\_generation(uni\_A, uni\_B, r\_A, c\_A) \\
\indent \indent \indent swap(uni\_A, uni\_B) \\
\indent \indent END FOR \\
\indent \indent endwin() \\
\indent ELSE \\
\indent \indent FOR k = 0 to num-1 \\
\indent \indent \indent one\_generation(uni\_A, uni\_B, r\_A, c\_A) \\
\indent \indent \indent swap(uni\_A, uni\_B) \\
\indent \indent END FOR \\
\indent END IF \\
\\
\indent IF out == FALSE \\
\indent \indent uv\_print(uni\_A, stdout) \\
\indent ELSE \\
\indent \indent foutptr $\leftarrow$ fopen(output, "w")) \\
\indent \indent uv\_print(uni\_A, foutptr) \\
\indent END IF \\
\\
\indent uv\_delete(uni\indent\_A) \\
\indent uv\_delete(uni\indent\_B) \\
\\
\indent RETURN 0 \\
End function \\
\\
\section{Credit}

\begin{itemize}
  \item Dev's section on Feb. 7th.
  \item The \textit{readfile\_example.c} in resources
  \item Modular arithmetic concepts from the \href{https://stackoverflow.com/questions/11720656/modulo-operation-with-negative-numbers}{discussion}
\end{itemize}
\end{document}